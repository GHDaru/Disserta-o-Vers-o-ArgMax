\section{Código Fonte para Geração das Matrizes Termos Documento}
\label{appendix:codigo-fonte-matrizes}
A seguir, apresentamos o código fonte utilizado para a geração das matrizes de frequência, TF-IDF e TF-IDF normalizado, juntamente com comentários explicativos:

\begin{verbatim}
from sklearn.feature_extraction.text import CountVectorizer, TfidfVectorizer
from sklearn.preprocessing import normalize
import pandas as pd

# Definindo os textos para as classes ARROZ e FEIJÃO
textos_arroz = ["arroz tio joao 1 kg arroz fumacence parb 1kg"]
textos_feijao = ["feijao carioca 1kg azulao feij 1 kg preto caldao"]

# Unindo os textos das duas classes para análise conjunta
textos = textos_arroz + textos_feijao

# Criando o vetorizador para contagem de frequência dos termos
count_vectorizer = CountVectorizer(token_pattern=r"(?u)\b\w+\b")
X_count = count_vectorizer.fit_transform(textos)

# Criando o vetorizador TF-IDF para calcular a frequência ponderada dos termos
tfidf_vectorizer = TfidfVectorizer(token_pattern=r"(?u)\b\w+\b")
X_tfidf = tfidf_vectorizer.fit_transform(textos)

# Normalizando os valores TF-IDF para obter uma distribuição mais uniforme
X_tfidf_norm = normalize(X_tfidf, norm='l1', axis=1)

# Convertendo os resultados para DataFrames para facilitar a visualização
df_count = pd.DataFrame(X_count.toarray(), 
                        columns=count_vectorizer.get_feature_names_out(), 
                        index=['ARROZ', 'FEIJÃO'])
df_tfidf = pd.DataFrame(X_tfidf.toarray(), 
                        columns=tfidf_vectorizer.get_feature_names_out(), 
                        index=['ARROZ', 'FEIJÃO'])
df_tfidf_norm = pd.DataFrame(X_tfidf_norm.toarray(), 
                             columns=tfidf_vectorizer.get_feature_names_out(), 
                             index=['ARROZ', 'FEIJÃO'])

# Definindo os índices multiníveis para as colunas
metodos = ['Frequência', 'TF-IDF', 'TF-IDF Normalizado']
termos = ['ARROZ', 'FEIJÃO']
colunas = pd.MultiIndex.from_product([metodos, termos], names=['Método', 'Termo'])

# Combinando os DataFrames com índices multiníveis
df_final = pd.concat([df_count.T, df_tfidf.T, df_tfidf_norm.T], axis=1)
df_final.columns = colunas

# Exibindo a tabela em formato LaTeX
print(df_final.to_latex())
\end{verbatim}
\end{table}
